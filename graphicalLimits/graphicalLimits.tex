\documentclass{ximera}
\input{../preamble}

\outcome{Find limits using graphs}

\title{Graphical Limits}


\begin{document}

\begin{abstract}
In this section we use the graph of a function to find limits.
\end{abstract}

\maketitle

Setting up github on new machine. New comment.
This section is under construction.
The graphs below are made with desmos.
The interactive graph at the very bottom is from geogebra.








\begin{center}
\bf{Finding Limits Graphically}
\end{center}

In this section, functions will be presented graphically. Recall that the graph of a 
function must pass the {\bf vertical line test}.  That is, a vertical line 
can intersect the graph of a function in at most one point.
To understand graphical representations of functions, consider the following graph of a function, $f$.

\[
\graph{y=(x^3 +7x^2)/6}
\]


%\begin{center}
%\includegraphics[width = 4 in, height = 3 in, angle= 1]{function01}
%\end{center}

The point $(-1,2)$ is on the graph of $f$.  This means that $f(-1) = 2$.  
Similarly, since the points $(0,0)$ and $(2,6)$ is on the graph, we have $f(0) = 0$ and $f(2) = 6$.  
The rationale is that the graph is created by letting the $x$-coordinate represent 
the input of the function and the $y$-coordinate represent the output of the 
function, i.e., $y = f(x)$.  The general form of a point on the graph of $y = f(x)$ is 
$(x_0, f(x_0))$ for any input value, $x_0$, in the {\bf domain} of $f$.


We now consider several examples of limits of functions presented graphically.




\begin{example} %example 1

For the function, $f$, whose graph is shown below, determine the value of the {\bf one-sided limit}, 
$\lim_{x\to 2^-} f(x)$. Note that this is called a one-sided limit because $x$ is approaching $2$ from the left hand side.


\[
\graph{2^x \left\{ -1 \leq x \leq 1.97 \right\}, (x-2)^2 + (y-4)^2 = 0.01}
\]


%\begin{center}
%\includegraphics[width = 4 in, height = 2.5 in]{graphical-limit01}
%\end{center}



Solution: Since $x \to 2^-$ we know that the $x$-value is approaching $2$ and we also know that $x<2$. 
Looking at the graph, we can see that for $x$-values slightly less than $2$ (to the left of 2), 
the $y$-values on the graph are very close to $4$. Thus, 

$$\lim_{x\to 2^-} f(x) = 4.$$

In this example, it is important to note that the open circle at the point $(2,3)$ indicates that $f(2)$ is undefined.
Thus, a function can have a limit as $x$ approaches a value that is not in the domain of the function. 

\end{example}

\begin{example} %example 2
For the function, $f$, whose graph is shown below, determine the value of the {\bf two-sided limit}, 
$\lim_{x\to 3} f(x)$.  Note that this is called a two-sided limit because $x$ can 
approach $2$ from either the left hand side or the right hand side.

%\begin{center}
%\includegraphics[width = 4 in, height = 3 in]{graphical-limit02}
%\end{center}


Solution: Since $x \to 3$ we know that the $x$-value is approaching $3$ and we also know 
that $x$ can be on either side of $3$ (but not equal to 3). 
Looking at the graph, we can see that for $x$-values either slightly less than $3$ or slightly greater than $3$, 
the $y$-values on the graph are very close to $2$. Thus, 

\[
\lim_{x\to 3} f(x) = 2.
\]

In this example it is important to observe that, even though the function value at $x= 3$ is $4$, 
as indicated by the dot at the point $(3,4)$, the limit of the function as $x$ approaches $3$ is $2$ and not $4$.  
The limit of a function is determined by the behavior of the function {\it near} the 
indicated $x$-value and not {\it at} that $x$-value.\\

\end{example}


\begin{example} %example 3
For the function, $f$, whose graph is shown below, find the one-sided limits,
$\lim_{x \to 1^-}f(x)$, $\lim_{x \to 1^+}f(x)$ and discuss the two-sided limit, $\lim_{x \to 1} f(x)$.

\[
\graph{(1,2),
2^x -3 \left\{-10 \leq x \leq 0.94 \right\}, 
-1/x +3 \left\{1 < x \leq 10 \right\}, (x-1)^2 + (y+1)^2 = 0.01,
}
\]


%\begin{center}
%\includegraphics[width = 4 in, height = 3 in]{graphical-limit03}
%\end{center}


Solution: As in example 1, we can determine the value of the left hand limit, $\lim_{x \to 1^-}f(x)$, 
by observing that if $x$ is on the left side of 1 and very close to 1 then the $y$-values on the 
graph are very close to -1 and hence, 

\[
\lim_{x \to 1^-}f(x) = -1.
\]

For the right hand limit $\lim_{x \to 1^+}f(x)$, we inspect the $y$-values on the 
graph that correspond to $x$-values on the right side of 1 and very close to $1$.  
We see that these $y$-values are very close to 2, hence

\[
\lim_{x \to 1^+}f(x) = 2.
\]

We see that the one-sided limits as $x$ approaches 1 have different values, and we therefore 
conclude that the two-sided limit, $\lim_{x \to 1}f(x)$, {\bf does not exist}.  We write

\[
\lim_{x \to 1}f(x) \;\;\mbox{DNE}.
\]

\end{example}

\begin{example} %example 4
For the function, $f$, whose graph is shown below, find the one-sided limit,
$\lim_{x \to 3^+}f(x)$.

\[
\graph[xmin=2, xmax=6, ymin=-10, ymax=3]{log(x-3)}
\]

%\begin{center}
%\includegraphics[width = 4 in, height = 2.5 in]{graphical-limit04}
%\end{center}


Solution: From the graph, we can see that as the $x$-values approach 3 from the right hand side, the $y$-values decrease without bound.  In this case we write

\[
\lim_{x \to 3^+}f(x) = -\infty.
\]

To describe the phenomenon of an {\bf infinite limit} as $x$ approaches a finite value, 
we say that the line $x = 3$ is a {\bf vertical asymptote} for the graph of $f$.

\end{example}

\begin{example} %example 5
For the function, $f$, whose graph is shown below, find the limit,
$\lim_{x \to \infty}f(x)$. This is an example of a {\bf limit at infinity}.

\[
\graph[xmin=0, xmax=10, ymin=-1, ymax=5]{(4/pi) arctan(x)}
\]

%\begin{center}
%\includegraphics[width = 4 in, height = 3 in, angle= -3]{graphical-limit05}
%\end{center}


Solution:  Since $x$ is approaching infinity, we look for a pattern on the right end of the graph.  
From the graph, we can see that the $y$-values are getting closer and closer to $2$ as the $x$ values 
increase without bound. We can conclude that 

\[
\lim_{x \to \infty}f(x) = 2,
\]

and we say that the line $y = 2$ is a {\bf horizontal asymptote} for the graph of $f$.


\end{example}

random graphs and such:

From Desmos

\[
\graph{(-3,1), (-2, 1), (-1, 1), (-1, 2),  (-1, 3), (-2, 2), (-3, 2),y=x^2}
\]

\[
\graph{2^x -3 \left\{ -10 \leq x \leq 1 \right\}, x^2 \left\{1 \leq x < 2 \right\}, x/4 \left\{ 2 \leq x < 3\right\}, 1/x \left\{3 \leq x \leq 10 \right\}}
\]

\[
\graph{(1,2), x^2}
\]


%fiddling around with sage

From Sage
\begin{sageOutput}
    show(diff(x^2))
\end{sageOutput}

From Geogebra
\geogebra{J4fcjvP9}{640}{480}




\end{document}